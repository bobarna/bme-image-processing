


%---Zoli's work-------------------


In one of the highly rated methods according to \cite{survOnMet} is based on the \gls{yolo} detector.
This \gls{alpr} system is introduced in \cite{DBLP:journals/corr/abs-1909-01754}. \gls{yolo} was first published in \cite{redmon2016look} as a novel object detection method.
The name is coming from the fact that it performs the detection by applying a single neural network to the full image.
This network divides the image into regions and predicts bounding boxes and probabilities for each region. 
Building on this method the authors achieved a license plate recognition rate of $96.9\%$.
The method was tested on multiple different datasets with outstanding results. 
Besides the novel approach the authors of this work also contributed by manually labelling $38,351$ bounding boxes on $6,239$ images from public datasets.
They made these annotations available to the public.

A benchmark for \gls{alpr} is showcased in \cite{DBLP:journals/corr/GoncalvesSMS16}. More specifically this benchmark is composed of a dataset helping the \gls{lpcs} step. High success rate of this step is crucial for end-to-end success of \gls{alpr}. In this paper besides the dataset the authors are proposing a new evaluation measure regarding the location of the bounding box within the ground-truth annotation. To further optimise the \gls{lpcs} step they suggest a more straightforward approach to perform it efficiently.









