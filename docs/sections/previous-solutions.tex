
%---Section intro-------------------

\ac{alpr} has been in the focus of many researches in the past years.
We have surveyed some of the most recent and most promising works in this field.
Our research was simplified due to a recently published survey paper \cite{survOnMet}.
This publication is giving an exhaustive list of different methods and practices used in \ac{alpr}.
We would like to mention some of the techniques stated in the said paper.

%---Gergő's work-------------------

Texture-based methods use the presence of characters the license plate is the basis for license plate recognition. Significant color difference between the board and its characters, creates a frequent color transition. Therefore, if the image is grayscale, there is a remarkable difference between the characters and the background of the board. This creates a unique pixel intensity distribution in the region of the plate. Plate region should have a high edge density. This is used in edge-based systems. In \cite{HongFuJiaHuan} the authors used scan-line technique for license plate detection.

%---Zoli's work-------------------


In one of the highly rated methods according to \cite{survOnMet} is based on the
\ac{yolo} detector.  This \ac{alpr} system is introduced in
\cite{DBLP:journals/corr/abs-1909-01754}. \ac{yolo} was first published in
\cite{redmon2016look} as a novel object detection method.  The name is coming
from the fact that it performs the detection by applying a single neural network
to the full image.  This network divides the image into regions and predicts
bounding boxes and probabilities for each region.  Building on this method the
authors achieved a license plate recognition rate of $96.9\%$.  The method was
tested on multiple different datasets with outstanding results.  Besides the
novel approach the authors of this work also contributed by manually labelling
$38,351$ bounding boxes on $6,239$ images from public datasets.  They made these
annotations available to the public.

A benchmark for \ac{alpr} is showcased in \cite{DBLP:journals/corr/GoncalvesSMS16}. More specifically this benchmark is composed of a dataset helping the \ac{lpcs} step. High success rate of this step is crucial for end-to-end success of \ac{alpr}. In this paper besides the dataset the authors are proposing a new evaluation measure regarding the location of the bounding box within the ground-truth annotation. To further optimise the \ac{lpcs} step they suggest a more straightforward approach to perform it efficiently.









