\section{Previous Solutions}
\label{previous-solutions}

\subsection{Automatic Number-Plate Recognition}
\label{previous-solutions-anpr}
% Section label
In the past years many research projects have been focused on \ac{ALPR}.
We have surveyed some of the most recent and most promising works in this field.
A recent survey by \cite{survOnMet} gives an overview of different methods and
practices used in \ac{ALPR}.

Texture-based methods use characters present on the \ac{LP} as the basis for
\ac{ALPR}. Significant color difference between the board and its characters
creates a high-frequency color transition. If the image is grayscale,
there is an easy to distinguish change of colors between the characters and the
background of the board. This creates a unique pixel intensity distribution in
the region of the plate. The plate region should have a high edge density. This
is used in edge-based systems. In \cite{HongFuJiaHuan} the authors used
scan-line technique for \ac{ALPR}.

Introduced by \cite{redmon2016look} as a novel object detection method,
\ac{YOLO} serves as the basis for the \ac{ALPR} introduced by
\cite{DBLP:journals/corr/abs-1909-01754}.
The naming of \ac{YOLO} comes from the fact that it performs the object detection
for the full image in a single pass. The employed \ac{NN} divides the image
into regions and predicts bounding boxes and probabilities for each region.
Building on this method the authors achieved a license plate recognition rate of
$96.9\%$.  The method was tested on multiple different datasets with outstanding
results.  Besides the novel approach, the authors of this work also released
a public dataset of $38,351$ manually labeled bounding boxes on $6,239$ images.

A benchmark for \ac{ALPR} is introduced by
\cite{DBLP:journals/corr/GoncalvesSMS16}. This benchmark is
composed of a dataset helping the \ac{LPCS} step. High success rate of this step
is crucial for end-to-end success of \ac{ALPR}. Besides the dataset, the authors
also propose a new evaluation measure of the location of the bounding
box within the ground-truth annotation. To further optimise the \ac{LPCS} step,
they suggest a more straightforward approach to perform it efficiently.

\subsection{Optical Character Recognition}
\label{previous-solutions-ocr}
\ac{DCNN}-s are 
Shi et al. \cite{7801919} proposed an architecture for automatic character recognition.


